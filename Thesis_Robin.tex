 % Created by Robin Sinurat (Modified from LaTeX Arini)
% The University Center of Excellence for Advanced Intelligent Communications (AICOMS),
% Telkom University, Bandung, INDONESIA
% 26 March 2022

% Ukuran font 12pt
\documentclass[12pt, a4paper, onecolumn, oneside, final]{report}
% Load konfigurasi LaTeX untuk tipe laporan thesis
\usepackage{ta}
\usepackage{longtable}
\usepackage{tikz}
\usepackage{graphics}
\usepackage{subcaption}
\usepackage{times}
\usepackage{graphicx}
\usepackage[square,numbers,comma,sort,compress]{natbib}\usepackage{titlesec}
\usepackage{amsmath}
\usepackage{appendix}
\onehalfspacing
\usepackage{hyperref} 
\usepackage{xcolor,colortbl}
\usepackage{multicol}
\usepackage[graphicx]{realboxes}
\usepackage[nodayofweek]{datetime}
\usepackage{multirow}

\renewcommand{\contentsname}{CONTENTS}
\renewcommand{\chaptername}{CHAPTER}
\renewcommand{\listfigurename}{LIST OF FIGURES}
\renewcommand{\listtablename}{LIST OF TABLES}
\renewcommand{\appendixname}{APPENDIX}
\renewcommand{\bibname}{REFERENCES}
\renewcommand{\figurename}{Fig.}

\usepackage[labelsep=space]{caption} % to remove colon after figure
%\captionsetup{labelsep = period}
\captionsetup{labelsep=separatorbaru}
%\usepackage{array}
\newcolumntype{L}[1]{>{\raggedright\let\newline\\\arraybackslash\hspace{0pt}}m{#1}}
\newcolumntype{C}[1]{>{\centering\let\newline\\\arraybackslash\hspace{0pt}}m{#1}}
\newcolumntype{R}[1]{>{\raggedleft\let\newline\\\arraybackslash\hspace{0pt}}m{#1}} % to remove colon after figure

% Load konfigurasi khusus untuk laporan yang sedang dibuat
%-----------------------------------------------------------------------------%
% Informasi Mengenai Dokumen
%-----------------------------------------------------------------------------%
% 
% Judul laporan. 
\var{\judul}{\setstretch{1}Babibu}
% 
% Tulis kembali judul laporan, kali ini akan diubah menjadi huruf kapital
\Var{\Judul}{\setstretch{1}Deformable Convolution Network and Deformable Attention in YOLO11 for Oriented Small Object Detection}

% Tipe laporan, dapat berisi Skripsi, Tugas Akhir, Thesis, atau Disertasi
\var{\type}{Master Thesis}
% 
% Tulis kembali tipe laporan, kali ini akan diubah menjadi huruf kapital
\Var{\Type}{Master Thesis}
% 

% Tulis nama penulis 
\Var{\Penulis}{Reynaldhi Tryana Graha}
% Tulis NPM penulis
\var{\nim}{201012420030}
\var{\alamat}{Rancasari, Bandung, Jawa Barat}
\var{\tlp}{+6285156964564}
\var{\email}{\textit{reynaldhitryanagraha@student.telkomuniversity.ac.id}}

% Tuliskan Fakultas dimana penulis berada
\Var{\Fakultas}{School of Electrical Engineering}
\var{\fakultas}{School of Electrical Engineering}
% 
% Tuliskan Program Studi yang diambil penulis
\Var{\Program}{Master of Electrical-Telecommunication Engineering}
\var{\program}{Master of Electrical-Telecommunication Engineering}
% 
% Tuliskan tahun publikasi laporan
\Var{\Tahun}{2025}
 
% Tuliskan tanggal pengesahan laporan, waktu dimana laporan diserahkan ke penguji/sekretariat
\var{\tanggalPengesahan}{\today} 

% Tuliskan pembimbing 
\var{\pembimbingSatu}{Dr. Koredianto Usman, S.T., M.Sc.}
\var{\nikSatu}{02750053}
\var{\pembimbingDua}{Suryo Adhi Wibowo, S.T., M.T., Ph.D.}
\var{\nikDua}{1087003}
% 
% Alias untuk memudahkan alur penulisan paa saat menulis laporan
\var{\saya}{Author}

%-----------------------------------------------------------------------------%
% Judul Setiap Bab
%-----------------------------------------------------------------------------%
% 
% Berikut ada judul-judul setiap bab. 
% Silahkan diubah sesuai dengan kebutuhan. 
% 
\var{\kataPengantar}{PREFACE}
\var{\babSatu}{INTRODUCTION}
\var{\babDua}{BASIC CONCEPT}
\var{\babTiga}{SYSTEM DESIGN AND MODEL}
\var{\babempat}{PERFORMANCE EVALUATIONS}
\var{\babLima}{CONCLUSION}


% Awal bagian penulisan laporan
\begin{document}

% Sampul Laporan
\begin{titlepage}
    \begin{center}      
    	\vspace*{0.15cm}
        % judul thesis harus dalam 14pt Times New Roman
        \bo{\large\Judul} \\[0.5cm]
        \vspace*{0cm}    
        % harus dalam 14pt Times New Roman
        \textbf{A MASTER'S THESIS}\\
        \vspace*{0.5cm}
		\setstretch{1}
		Submitted to\\
		Graduate School of Electrical Engineering
        \vspace*{1.5cm}
                      
        \begin{figure}
            \begin{center}
                \includegraphics[scale=0.35]{logo/logo_telu1}
            \end{center}
        \end{figure}
        \vspace*{0.75cm}
           % penulis dan npm
        By\\
        \vspace*{0.25cm}
        \bo{\Penulis} \\
        \vspace*{0.25cm}
        \bo{\nim} \\
        
        \vspace*{1.5cm}
        
        \setstretch{1}
        In partial fulfillment of the requirements\\
        for the Degree of Master of Engineering\\
        \vspace*{1.5cm}
        % informasi mengenai fakultas dan program studi
        \bo{
        	TELKOM UNIVERSITY\\
        	BANDUNG \\
        	\Tahun
        }
    \end{center}
\end{titlepage}


% Tidak mengaktifkan penomoran halaman
\pagenumbering{gobble}

% Halaman pengesahan
\addChapter{APPROVAL PAGE}
%-----------------------------------------------------------------------------
% Halaman Pengesahan
%-----------------------------------------------------------------------------


\begin{center}
	
	\textbf{\large{APPROVAL PAGE}}
	\vspace*{1 cm}
	
    \textbf{\large{MASTER'S THESIS}}
	\vspace*{1 cm}

    \textbf{\Judul}
%	\vspace*{1 cm}    
%    
%    \textit{\textbf{\JudulInggris}}
	\vspace*{1.5 cm}
    
    by\\
    \vspace*{0.25 cm}
    
    \textbf{\Penulis}\\
    \textbf{\nim}\\
   	\vspace*{2 cm}   
    
     {\bfseries Approved and authorized to fulfil one of the requirements of\\Program of \program\\ \fakultas\\ Telkom University\\Bandung
    }
 %    \\
	% \textbf{Program of \program}\\
 %    \textbf{\fakultas}\\
 %    \textbf{Telkom University}\\
 %    \textbf{Bandung}
 
    
    
    \vspace*{3 cm}
    
    \textbf{Bandung, \today}\\
%    \textbf{Menyetujui,}
    
\end{center}
    
\begin{tabular}
	{>{\centering\arraybackslash} p{0.3\paperwidth} >{\centering\arraybackslash} p{0.34\paperwidth}}\\
    Supervisor & Co-Supervisor\\[3 cm]
    \uline{\textbf{\pembimbingSatu}} & \uline{\textbf{\pembimbingDua}}\\
    NIP. \nikSatu & NIP. \nikDua
\end{tabular}

%-----------------------------------------------------------------------------

% Halaman pernyataan orisinalitas
\addChapter{SELF DECLARATION AGAINST PLAGIARISM}
\chapter*{}

%\AddToShipoutPicture*{%
%\gradientbox{white}{white}{%
%
%\begin{minipage}[ct][0.98\paperheight][t]{\paperwidth}%
%\end{minipage}}}
%
%    \begin{center}
%    \textbf{SELF DECLARATION AGAINST PLAGIARISM}\\
%    \end{center}
%    
%    \begin{tabular}{ll}
%    Nama & :\hspace*{0.2 cm}\penulis \\
%    NIM & :\hspace*{0.2 cm}\nim \\
%    Alamat & :\hspace*{0.2 cm}\alamat \\
%    No. Telepon & :\hspace*{0.2 cm}\tlp \\
%    Email & :\hspace*{0.2 cm}\email \\
%    \end{tabular}
%    
%    \vspace*{1 cm}
%    Menyatakan bahwa Tugas Akhir ini merupakan karya orisinal saya sendiri, dengan judul:
%    
%    \begin{center}
%    \textbf{\Judul}\\
%%    \textit{\textbf{\JudulInggris}}\\
%    \end{center}
%    
%    Atas pernyataan ini, saya siap menanggung resiko\slash sanksi yang dijatuhkan kepada saya apabila kemudian ditemukan adanya pelanggaran terhadap kejujuran akademik atau etika keilmuan dalam karya ini, atau ditemukan bukti yang menunjukkan ketidakaslian karya ini.
%    
%    \vspace*{1 cm}
%    
%    \begin{tabular}{cl}
%    \multirow{6}{*}{\includegraphics[scale=0.03,origin=c]{pics/fotoa}\hspace{4 cm}}
%    & Bandung, \tanggalPengesahan \\
%    & \includegraphics[scale=1,origin=c]{pics/sign_Arini} \\
%    & \\
%    & \penulis \\
%    \cline{2-2}
%    &  \nim\\
%    \end{tabular}
    
    \begin{center}
	\textbf{\large{SELF DECLARATION AGAINST PLAGIARISM}}
	\vspace*{1 cm}
\end{center}
	I hereby declare that all information in this document has been obtained and presented in accordance with academic rules and ethical conduct. I also declare that, as required by these rules and conduct. I have full cited and referenced all materials and results that are not original to this work.\\
	
	\noindent\today\\
	\Penulis\\
	\includegraphics[width=6cm]{signature/TTD REY.jpg}
	
	\noindent Signature:\rule{4cm}{0.4pt}


% Menggunakan penomoran halaman Romawi
\pagenumbering{roman}

% Setelah bagian ini, halaman dihitung sebagai halaman ke-2
\setcounter{page}{4}

% Abstrak dalam Bahasa Inggris
\addChapter{ABSTRACT}

\chapter*{abstract}
\vspace*{-0.3cm}

\indent \indent 
Quantum technology in recent decades is attracting great attentions, especially on the quantum computing, quantum communications, and quantum key distribution (QKD), which are candidates for the sixth generation of telecommunications (6G) in 2030 and beyond.

\vspace{0.2in}
\noindent\textbf{Keywords:} CSS, Decoherence, Depolarizing Channels, Red-Muller codes, QWER.


\newpage

% Ucapan Terima Kasih
\addChapter{ACKNOWLEDGEMENTS}
%-----------------------------------------------------------------------------%
\chapter*{ACKNOWLEDGEMENTS}
%-----------------------------------------------------------------------------%
This thesis is compiled with the effort, help, and support from all supporting elements. The author would like to express the deepest gratitude and thanks to:
\begin{enumerate}
\item Allah SWT, for all the love, guidance and forgiveness in every mistake that the author has ever done and Rasulullah SAW, as role model who inspire writer in living life and trying to be better.

\end{enumerate}
  
%\vspace*{0.1cm}
%\begin{flushright}
%Bandung, 02 Februari 2019\\[0.1cm]
%\vspace*{1cm}
%\penulis

%\end{flushright}

% Kata Pengantar
\addChapter{\kataPengantar}
%-----------------------------------------------------------------------------%
\chapter*{\kataPengantar}
%-----------------------------------------------------------------------------%
	Alhamdu lillahi rabbil 'alamin, praise to Allah, the most gracious, the most merciful, with the mercy and guidance, the author has successfully finished this thesis with the title of ~"\textbf{\Judul}". The author compiled this thesis to be filled in the graduation requirements in Program of Master of Electrical-Telecommunication Engineering, School of Electrical Engineering, Telkom University.
		
	Some parts of this thesis also have been submitted to The 9th Asia Pacific Conference on Wireless \& Mobile (APWIMOB) Conference 2024.
	
	The suggestions for improving this thesis are highly appreciated. Hopefully, this thesis is expected to be improved and provided contributions for the reader and Indonesia especially for education and research of telecommunication on the future.   

\vspace*{1 cm}
\begin{flushright}
Bandung, \tanggalPengesahan\\
\vspace*{2.75 cm}

\Penulis

\end{flushright}
%-----------------------------------------------------------------------------

% Daftar Isi
\tableofcontents
\clearpage

% Daftar Gambar
\listoffigures
\clearpage

% Daftar Tabel
\listoftables
\clearpage

% Daftar Lampiran
\addChapter{LIST OF ABBREVIATION}
%-----------------------------------------------------------------------------%
\chapter*{LIST OF ABBREVIATION}
%-----------------------------------------------------------------------------%
\begin{tabular}{lll}
1G &:& First Generation \\
2G &:& Second Generation \\
3G &:& Third Generation \\
4G &:& Fourth Generation \\
5G &:& Fifth Generation \\
6G &:& Sixth Generation \\
CSS &:& Calderbank-Shor-Steane\\
MDS &:& Maximum distance separable\\
LUT &:& Look Up Table\\
QECC &:& Quantum Error Correction Codes\\
QKD &:& Quantum Key Distribution\\
QWER &:& Quantum Word Error Rate\\
RM &:& Reed-Muller codes\\
SIP &:& Simplectectic Inner Product\\
WP &:& Work Packages\\

\end{tabular}

% Daftar Lampiran
\addChapter{LIST OF SYMBOL}
%-----------------------------------------------------------------------------%
\chapter*{LIST OF SYMBOL}
%-----------------------------------------------------------------------------%
\begin{longtable}{lll}
$A_{mqk}$         & Attention weight for the sampled location $(m,q,k)$ \\
$AP_i$            & Average Precision for class i \\
$B$               & Batch size \\
$B_i$             & Number of biases in layer i \\
$C$               & Number of channels in the feature map \\
$d_k$             & Dimension of the key vector \\
$\Delta m_n$      & Modulation scalar weighting each deformed sample \\
$\Delta p_{mqk}$  & Attention offset for $k^{th}$ key in head $m$ \\
$\Delta p_n$      & Deformable offset added to the sampling grid \\
$F_i$             & Number of floating-point operations in layer i \\
$FN$              & False Negatives \\
$FP$              & False Positives \\
$G(q, p)$         & Bilinear interpolation kernel \\
$H$               & Height of the feature map \\
$K$               & Number of sampling keys per attention head \\
$L$               & Number of layers \\
$M$               & Number of attention heads in the deformable module \\
$N$               & Number of classes (in mAP context) \\
$p_q$             & Reference point used by Deformable Attention for each query \\
$Q$               & Query matrix in attention mechanism \\
$\mathcal{R}$     & Regular sampling grid of a convolution kernel \\
$S$               & Stride of the feature map \\
$TP$              & True Positives \\
$V$               & Value matrix in attention mechanism \\
$W$               & Width of the feature map \\
$W_i$             & Number of weights in layer i \\
$W_m$             & Output projection matrix of attention head $m$ \\
$W'_m$            & Value projection matrix of attention head $m$ \\
$w(p_n)$          & Learned convolution weight at offset $p_n$ \\
$x(p)$            & Feature value sampled at position $p$ (possibly fractional) \\
$y(p_0)$          & Output feature at spatial location $p_0$ \\
$z_q$             & Content feature of the query element $q$ \\
\end{longtable}


% Daftar Achievement
\addChapter{ACHIEVEMENT}
%-----------------------------------------------------------------------------
% Daftar Achievement
%-----------------------------------------------------------------------------


\chapter*{Achievements}

\vspace*{0.5 cm}

\begin{enumerate}
\item AAAAA
\end{enumerate}

%-----------------------------------------------------------------------------

% Gunakan penomeran Arab (1, 2, 3, ...) setelah bagian ini.
\pagenumbering{arabic}

% Bab 1 : Pendahuluan
%-----------------------------------------------------------------------------%
\chapter{\babSatu}
%-----------------------------------------------------------------------------%
\vspace*{0.2cm}
\hyphenation{se-ve-ral}

\section{Background}
% Berisi rasional dari riset serta permasalahan yang diperkuat dengan sitasi dari literature (Paper conference/paper jurnal/textbook 3 tahun terakhir). Permasalahan dapat diambil dari penelitian/pekerjaan sebelumnya dalam 3 tahun terakhir. Latar belakang dibuat dalam 1-3 paragraf. Pada alinea terakhir, nyatakan kaitan antara Tesis yang akan dibuat dengan penelitian/pekerjaan sebelumnya. Berisi minimal 1.5 halaman. 

% Object Detection (General Context)
Object detection is a fundamental task in computer vision with numerous applications across various domains, including medical imaging~\cite{Sobek2024}, autonomous driving~\cite{Alahdal2024}, security surveillance~\cite{Abba2024}, and aerial imaging~\cite{Saini2025Report}. The primary objective of object detection is to identify and localize objects within an image or video frame by predicting bounding boxes and class labels for each detected object. Over the years, significant advancements have been driven by the development of deep learning algorithms, specifically Convolutional Neural Networks (CNNs). These advancements have led into two main categories of object detection methods: two-stage detectors and single-stage detectors. Two-stage detectors, such as Faster R-CNN, prioritize accuracy by first generating region proposals before classification~\cite{NIPS2015_14bfa6bb}. In contrast, single-stage detectors, like  You Only Look Once (YOLO) and Single Shot MultiBox Detector (SSD), optimize for speed by directly predicting bounding boxes and class probabilities in a single pass~\cite{redmon2016lookonceunifiedrealtime, Liu_2016}. The YOLO family, in particular, has undergone rapid evolution leading to the recent YOLO11, which offers a superior balance between inference speed and detection accuracy~\cite{khanam2024yolov11overviewkeyarchitectural}.

% Small Object Detection (Problem Statement 1)
Despite the impressive performance of single-stage detectors on general tasks, they frequently encounter difficulties when detecting small objects. Small objects typically defined as those occupying a minimal number of pixels present significant challenges due to limited visual information, susceptibility to occlusion, and background clutter~\cite{Nikouei_2025}. This issue is particularly critical in aerial imagery analysis, where targets such as vehicles, buildings, and ships appear at varying scales~\cite{YOLO-Air_10980347}. The primary challenges include the loss of critical features during CNN downsampling, the extreme class imbalance between small and large objects, and the difficulty in distinguishing small objects from noise. To mitigate these issues, researchers have explored techniques such as multi-scale feature fusion, context-aware modeling, and specialized data augmentation~\cite{Cheng_2023}. Additionally, architectural enhancements like Feature Pyramid Networks (FPNs) and attention mechanisms have shown promise in recovering details necessary for small object detection~\cite{app12188940}.

% Oriented Object Detection (Problem Statement 2)
Beyond scale, another critical aspect in aerial object detection is orientation. Unlike traditional detection methods that utilize axis-aligned Horizontal Bounding Boxes (HBB), oriented object detection employs Oriented Bounding Boxes (OBB) or polygons to capture objects more precisely. This capability is essential for aerial imagery, where objects often appear with arbitrary rotations due to the top-down perspective of the imaging sensor~\cite{YOLO-Air_10980347}. Standard HBBs often introduce excessive background noise when enclosing rotated objects, confusing the classifier. Consequently, oriented detection methods incorporate angle regression, rotation-invariant feature extraction, and specialized loss functions to effectively handle these geometric variations~\cite{yang2020r3detrefinedsinglestagedetector}.

% Limitations (The "Gap")
However, standard CNNs backbone of most modern detectors including YOLO possess inherent limitations when simultaneously addressing orientation and small scale. Standard convolution operations sample the input feature map using a fixed, regular grid. This rigid geometric structure lacks invariance to large geometric transformations such as rotation, scaling, or deformation~\cite{dai2017deformableconvolutionalnetworks}. Consequently, when a small object is rotated or appears in a non-standard pose, the fixed receptive field of a standard convolution may fail to cover the object of interest effectively~\cite{yuan2025empiricalstudymethodssmall, CHEN2024117194}. This limitation is particularly detrimental for small oriented objects, where semantic information is already scarce, and any misalignment in feature extraction can lead to detection failures~\cite{rekavandi2023transformerssmallobjectdetection}.


% Proposed Solution (The "Fix")
To overcome these geometric and spatial limitations, advanced mechanisms such as Deformable Convolution Networks (DCN) and Attention Mechanisms have been proposed. Deformable convolution introduces learnable offsets to the regular sampling grid, allowing the receptive field to adaptively deform and align with the object's actual shape and orientation~\cite{dai2017deformableconvolutionalnetworks, wang2023internimageexploringlargescalevision}. Furthermore, attention mechanisms, specifically Deformable Attention, enable the network to focus dynamically on the most relevant features while suppressing irrelevant background information~\cite{zhu2021deformabledetrdeformabletransformers}. By adjusting the importance of different spatial locations, these mechanisms enhance the representation of small objects that might otherwise be overwhelmed by background clutter~\cite{yuan2025empiricalstudymethodssmall}.

% Thesis Statement
Therefore, this research proposes the integration of Deformable Convolution and Deformable Attention mechanisms into the YOLO11 architecture to enhance feature extraction capabilities for oriented small object detection. By leveraging the adaptive sampling of deformable convolution and the context-aware focusing of deformable attention, the proposed method aims to address the geometric variations and feature scarcity inherent in aerial imagery. This thesis explores the synergistic effect of these components within the YOLO11 framework, aiming to achieve superior detection performance compared to existing state-of-the-art methods on challenging aerial and remote sensing datasets.

\section{Problem Identification}
% Berisi penjelasan permasalahan yang dihadapi untuk menyelesaikan penelitian, dalam 1 kalimat. Selanjutnya dapat dibuat dalam bentuk turunan masalah yang detail. Rumusan masalah ini berisi minimal ½ halaman.

The primary problem addressed in this research is the main two challenges of detecting objects that are both tiny in scale and arbitrarily oriented within aerial imagery, which modern detectors struggle to handle due to the fixed geometric structure of standard Convolutional Neural Networks (CNNs).

Specifically, the technical problems are identified as follows:

\begin{enumerate}
    \item \textbf{Geometric Rigidity of Standard Convolution:} Standard convolution operations rely on kernels with a fixed geometric shape. This rigid structure inherently restricts the network's ability to model complex geometric transformations. When objects in aerial imagery appear at arbitrary angles, the fixed receptive field of the convolution kernel fails to adapt, preventing the network from effectively capturing the features of rotated objects.
    
    \item \textbf{Feature Loss in Low-Resolution Targets:} Small objects in aerial imagery typically possess low resolution and limited feature information. As these images pass through the downsampling layers of standard CNN architectures, the already scarce spatial details are often lost or `washed out`. Consequently, the detector fails to preserve the fine-grained information necessary to recognize these very small targets in deeper network layers.
    
    \item \textbf{Interference from Background Noise:} Due to their small scale and limited features, small objects are difficult to distinguish from environmental background noise. Standard feature extraction mechanisms often lack the ability to focus exclusively on the object of interest, causing the weak feature signals of small targets to be overwhelmed by stronger signals from complex backgrounds, leading to missed detections.
    
    \item \textbf{Inefficiency of Horizontal Bounding Boxes:} Traditional object detection relies on Horizontal Bounding Boxes (HBB). For oriented objects, HBBs are inefficient because they inevitably capture excessive background information along with the object. This inclusion of irrelevant background noise within the object proposal confuses the classification process and hinders the precise localization required for aerial targets.
\end{enumerate}


\section{Research Objective}
% Berisi tujuan akhir penelitian atau objective penelitian yang akan dicapai serta dapat ditambahkan penjelasan lebih rinci atau ditulis lebih detail dalam bentuk point-point tujuan atau objective. Tujuan atau objective penelitian berisi minimal ½ halaman.

The primary objective of this research is to develop an enhanced hybrid single-stage object detection framework based on YOLO11, specifically optimized for the two main challenges of oriented small object detection. This is achieved by addressing the geometric rigidity of standard convolutions and the lack of context-aware focusing in current baselines through the integration of adaptive feature extraction mechanisms.

The specific objectives of this research are detailed as follows:

\begin{enumerate}
    \item \textbf{Integrate Deformable Convolution Modules for Geometric Adaptation:} Strategically modify the standard convolutional layers within the YOLO11 architecture by integrating Deformable Convolution modules. This modification aims to enable the network to adaptively adjust its receptive field based on the scale and orientation of target objects. By learning offset values for the sampling grid, the model will capture the geometric features of rotated objects more effectively than the fixed-grid convolutions used in the baseline model, directly addressing the issue of geometric rigidity.

    \item \textbf{Implement Deformable Attention for Robust Feature Fusion:} Enhancing the feature fusion process by incorporating Deformable Attention mechanisms. Unlike standard attention modules which often treat spatial locations uniformly, this mechanism is designed to focus computational resources sparsely on the most informative key points of small objects. This objective seeks to demonstrate that dynamic, sparse attention can effectively suppress the background clutter and noise inherent in aerial imagery, therefore preserving the weak feature signals of small targets.

    \item \textbf{Design and Validate a Hybrid Architecture While Balancing Accuracy and Speed:} Design and validate a cohesive hybrid architecture that synergizes Deformable Convolution and Deformable Attention to optimize Oriented Bounding Box (OBB) task. This objective involves benchmarking the proposed framework against state-of-the-art methods on large-scale aerial datasets (DOTA and SODA) to demonstrate a superior balance between detection accuracy (mAP) and computational efficiency (FLOPs), ensuring the model remains practical for real-world deployment.
\end{enumerate}

\section{Research Method}
% Berisi penjelasan singkat tentang metode/formula/skema/algoritma utama (1-2 metoda) yang akan digunakan/diusulkan dalam penelitian, berdasarkan referensi utama yang akan dijadikan acuan. Berisi minimal 1 halaman. 
This research proposes a novel single-stage object detection framework that enhances the YOLO11 architecture by integrating two advanced adaptive feature extraction mechanisms: Deformable Convolutional Networks (DCN) and Deformable Attention. These methods are selected to specifically address the two challenges of geometric variations in oriented objects and feature scarcity in small objects.

\subsection{Deformable Convolutional Networks (DCN)}
Standard Convolutional Neural Networks (CNNs) are inherently limited in modeling geometric transformations due to the fixed geometric structures of their building modules. A standard convolution unit samples the input feature map at fixed locations (e.g., a regular $3 \times 3$ grid) and pools features with a static receptive field. This rigidity is suboptimal for oriented object detection, where targets may appear with arbitrary rotation, scaling, or deformation.

To overcome this limitation, this research employs Deformable Convolution, as introduced by Dai et al.~\cite{dai2017deformableconvolutionalnetworks}. The core idea is to augment the spatial sampling locations in the convolution modules with learnable offsets.

In a standard 2D convolution, the output feature map $y$ at a specific location $p_0$ is computed as:

\begin{equation}
    y(p_0) = \sum_{p_n \in \mathcal{R}} w(p_n) \cdot x(p_0 + p_n)
\end{equation}

where $\mathcal{R}$ defines the regular sampling grid (e.g., $\mathcal{R} = \{(-1, -1), (-1, 0), \dots, (1, 1)\}$ for a $3 \times 3$ kernel) and $w(p_n)$ represents the weights.

In Deformable Convolution, the regular grid $\mathcal{R}$ is augmented with offsets $\{\Delta p_n | n=1, \dots, N\}$, where $N = |\mathcal{R}|$. The equation is reformulated as:

\begin{equation}
    y(p_0) = \sum_{p_n \in \mathcal{R}} w(p_n) \cdot x(p_0 + p_n + \Delta p_n)
\end{equation}

Here, the sampling is performed on the irregular and offset locations $p_n + \Delta p_n$. The offsets $\Delta p_n$ are obtained by applying a separate convolutional layer over the same input feature map, allowing the deformation to be conditioned on the input features in a local, dense, and adaptive manner.

Since the learned offset $\Delta p_n$ is typically fractional, the pixel value $x(p)$ at an arbitrary location $p = p_0 + p_n + \Delta p_n$ is computed via bilinear interpolation:

\begin{equation}
    x(p) = \sum_{q} G(q, p) \cdot x(q)
\end{equation}

where $q$ enumerates all integral spatial locations in the feature map $x$, and $G(\cdot, \cdot)$ is the bilinear interpolation kernel. This differentiability allows the offsets to be learned end-to-end via standard back-propagation.

\subsection{Deformable Attention}
While DCN improves geometric adaptability, detecting small objects in cluttered aerial images requires a mechanism to focus on sparse, informative key elements. Standard Transformer attention modules suffer from slow convergence and high computational complexity because they look over all possible spatial locations in the image feature maps.

To mitigate these issues, this research integrates the Deformable Attention module proposed by Zhu et al.~\cite{zhu2021deformabledetrdeformabletransformers}. This mechanism combines the sparse spatial sampling of deformable convolution with the relation modeling capability of Transformers.

Unlike standard Multi-Head Self-Attention (MHSA) which has quadratic complexity relative to pixel numbers, the Deformable Attention module only attends to a small set of key sampling points around a reference point.

Given an input feature map $x \in \mathbb{R}^{C \times H \times W}$, a query element $q$ with content feature $z_q$, and a 2-D reference point $p_q$, the Deformable Attention feature is calculated as:
\begin{equation}
    \text{DeformAttn}(z_q, p_q, x) = \sum_{m=1}^{M} W_m \left[ \sum_{k=1}^{K} A_{mqk} \cdot W'_m x(p_q + \Delta p_{mqk}) \right]
\end{equation}
where:
\begin{itemize}
    \item $m$ indexes the attention head ($M$ heads total).
    \item $k$ indexes the sampled keys ($K$ keys total), where $K \ll HW$.
    \item $\Delta p_{mqk}$ and $A_{mqk}$ denote the sampling offset and attention weight of the $k^{th}$ sampling point, respectively.
    \item $A_{mqk}$ is normalized such that $\sum_{k=1}^{K} A_{mqk} = 1$.
\end{itemize}

Similar to DCN, the term $x(p_q + \Delta p_{mqk})$ is computed using bilinear interpolation. This design allows the model to focus on a small, fixed number of keys for each query, significantly reducing computational complexity to $O(2N_q C^2 + \min(HWC^2, N_q K C^2))$, which is linear with respect to the spatial size.

\section{Hypothesis}
% Berisi prediksi hasil dan dasar prediksi yang digunakan berdasarkan referensi hasil pekerjaan/penelitian sebelumnya. Sertakan referensi yang disitasi untuk memprediksi hasil. Hipotesis Berisi minimal ½ halaman.

Based on the structural limitations of standard CNNs and the proposed architectural enhancements, this research posits the following hypotheses regarding the performance of the modified YOLO11 framework:

\begin{enumerate}
    \item \textbf{Improved Accuracy for Oriented Objects:} Integrating Deformable Convolution modules into the YOLO11 backbone will significantly improve the detection accuracy of oriented objects compared to the baseline model. This prediction relies on the ability of Deformable Convolution to learn adaptive sampling offsets, allowing the receptive field to align dynamically with the rotation and geometric variations of aerial targets~\cite{dai2017deformableconvolutionalnetworks}.

    \item \textbf{Enhanced Small Object Detection:} Implementing Deformable Attention mechanisms will measurably increase the detection performance for small objects in cluttered environments. By focusing computational resources on a sparse set of key sampling points rather than the entire feature map, this mechanism is expected to suppress background noise and preserve the weak feature signals of small targets more effectively than the baseline architecture~\cite{zhu2021deformabledetrdeformabletransformers}.

    \item \textbf{Synergistic Performance of the Hybrid Architecture:} The hybrid combination of Deformable Convolution and Deformable Attention is expected to yield a synergistic effect, achieving a superior balance between accuracy (mAP) and computational efficiency. The proposed architecture is expected to outperform both the baseline YOLO11 and single-modification variants on the DOTA and SODA datasets by simultaneously addressing geometric misalignment and feature scarcity without incurring the quadratic complexity of standard Transformers.
\end{enumerate}

\section{Research Methodology}
% Berisi urutan langkah – langkah untuk menyelesaikan penelitian beserta teknik/metoda disetiap langkah. Buat dalam bentuk diagram blok dan deskripsikan setiap langkah tersebut. Berisi minimal 1 halaman.
The research methodology employed in this thesis follows a systematic experimental approach, structured around distinct Work Packages (WP). This structure ensures a logical progression from theoretical understanding to practical implementation and final evaluation. The overall flow of the research is visualized in Figure~\ref{fig:research_flowchart}.

\begin{figure}
    \centering
    \includegraphics[width=1.0\textwidth]{pictures/Proposal Thesis - Research Methodology Diagram.png} 
    \caption{Research Methodology Flowchart}
    \label{fig:research_flowchart}
\end{figure}

The specific Work Packages (WP) for this research are outlined as follows:

\begin{itemize}
    \item \textbf{WP 1: Literature Review} \\
    This work package involves an in-depth analysis of existing literature to establish a strong theoretical foundation. The review focuses on the evolution of the YOLO architecture up to YOLO11, the mathematical principles of Deformable Convolutional Networks (DCN), and the mechanics of Deformable Attention.

    \item \textbf{WP 2: Dataset Acquisition and Preprocessing} \\
    This work package focuses on preparing the data required for training and evaluation. Acquiring standard aerial imagery datasets such as DOTA (Dataset for Object deTection in Aerial images) and SODA (Small Object Detection dAtaset). Then preprocess data annotation labels from polygon or complex formats into the YOLO OBB format (center-point, width, height, angle).

    \item \textbf{WP 3: Architectural Design and Modification} \\
    This is the core development phase where the YOLO11 baseline is structurally enhanced. The modifications include:
    \begin{enumerate}
        \item \textbf{DCN Integration:} Replacing standard convolutional layers in the Backbone and Neck with Deformable Convolution modules to enable adaptive receptive field learning.
        \item \textbf{Attention Integration:} Designing the replacement of the standard C2PSA block with the Deformable Attention mechanism to improve feature focusing on small targets.
        \item \textbf{Hybrid Design:} Validating the two methods compatibility and ensuring that the combined architecture maintains computational efficiency while enhancing detection capabilities.
    \end{enumerate}

    \item \textbf{WP 4: Model Training and Performance Evaluation} \\
    This work package include the implementation, training, and quantitative evaluation of the proposed architecture. The model is implemented in PyTorch within Ultralytics YOLO framework. Then the model is then trained and evaluated on a unseen test set using standard metrics, including Mean Average Precision (mAP) for Oriented Bounding Boxes (mAP50 and mAP50-95). Furthermore, computational efficiency is benchmarked via Number of Parameters (Params) and Floating Point Operations (FLOPs) to ensure practical viability against the baseline YOLO11.

    \item \textbf{WP 5: Ablation Study and Analysis} \\
    To validate the individual contributions of the proposed enhancements, this WP involves conducting ablation studies. Separate models will be trained with single modifications (e.g., YOLO11+DCN only, YOLO11+Deformable Attention only) to isolate the performance gains attributed to Deformable Convolution versus Deformable Attention. The analysis will also include qualitative visualization of detection results to identify improvements in handling background clutter and rotation.
\end{itemize}

\section{Timeline}
% Bagian ini berisikan jadwal pengerjaan. Jadwal pengerjaan merupakan jadwal kegiatan penelitian, diantaranya studi literatur, perancangan desain sitem dan model, simulasi, implementasi rancangan, pembuatan prototype, sampai dengan pengujian dan analisis, termasuk membuat luaran penelitian dalam bentuk publikasi. Jadwal dapat ditulis dalam bentuk tabel. Tingkat kedetilan jadwal dapat dalam bulan maupun minggu, dengan rentang waktu sesuai dengan kebutuhan dan target.
The research activities are scheduled to be completed within a period of six months. The timeline is structured to ensure that each phase of the methodology is given sufficient time for thorough execution and analysis. Table~\ref{tab:timeline} details the schedule of activities.

\begin{table}
    \centering
    \caption{Research Timeline}
    \label{tab:timeline}
    \begin{tabularx}{\textwidth}{|c|X|*{6}{c|}} % chktex 44
        \hline % chktex 44
        \multirow{2}{*}{\textbf{No.}} & \multicolumn{1}{c|}{\multirow{2}{*}{\textbf{Activity}}} & \multicolumn{6}{c|}{\textbf{Month}} \\ \cline{3-8} 
         & & \textbf{1} & \textbf{2} & \textbf{3} & \textbf{4} & \textbf{5} & \textbf{6} \\ \hline % chktex 44
        1 & Literature Review & \cellcolor{green!50} & & & & & \\ \hline % chktex 44
        2 & Dataset Acquisition and Preprocessing & \cellcolor{green!50} & \cellcolor{green!50} & & & & \\ \hline % chktex 44
        3 & Architectural Design and Modification & & \cellcolor{green!50} & \cellcolor{green!50} & & & \\ \hline % chktex 44
        4 & Model Training and Performance Evaluation & & & \cellcolor{green!50} & \cellcolor{green!50} & \cellcolor{green!50} & \\ \hline % chktex 44
        5 & Ablation Study and Analysis & & & & & \cellcolor{green!50} & \cellcolor{green!50} \\ \hline % chktex 44
        6 & Conclusion and Thesis Writing & & & & & & \cellcolor{green!50} \\ \hline % chktex 44
    \end{tabularx}
\end{table}

\begin{itemize}
    \item \textbf{Month 1:} Focus on understanding the theoretical background, reviewing related works, and finalizing the research proposal. Initial dataset acquisition begins.
    \item \textbf{Month 2:} Completion of dataset preprocessing. Start of the architectural design phase, identifying where to integrate DCN and Attention modules.
    \item \textbf{Month 3:} Finalizing the system design. Beginning the implementation of the modified YOLO11 model in code and starting initial training runs.
    \item \textbf{Month 4:} Intensive model training and tuning. Conducting the first round of evaluations on the validation set.
    \item \textbf{Month 5:} Performing comprehensive testing on the test set. Conducting ablation studies to isolate the effects of specific modules. Beginning the analysis of results.
    \item \textbf{Month 6:} Finalizing the analysis. Writing the complete thesis report, preparing for the defense, and drafting a paper for publication.
\end{itemize}

%
% Bab 2 : Dasar Teori
%----------------------------------------------------------------------
\chapter{\babDua}
%--------------------------------------------------------
This chapter establishes the theoretical and architectural foundations requisite for understanding the proposed enhancements to the YOLO11 framework. It provides an exhaustive examination of the evolution of object detection paradigms, delineating the unique pathological challenges associated with detecting small, arbitrarily oriented objects in aerial imagery. Furthermore, this chapter presents a rigorous mathematical treatment of Convolutional Neural Networks (CNNs), the architectural specifics of YOLO11, and the adaptive mechanisms of Deformable Convolution Networks (DCN) and Deformable Attention, which constitute the core technical contributions of this thesis.

\section{Object Detection}
Object detection stands as one of the most fundamental and challenging problems in the field of computer vision. Unlike image classification, which assigns a single label to an entire image, or semantic segmentation, which classifies pixels without differentiating object instances, object detection requires the simultaneous localization and classification of multiple objects within a scene \cite{zou2023object, liu2020deep}. The primary objective is to predict a set of bounding boxes—rectangular delineations of object boundaries—and associate a class probability score with each box. The complexity of this task is compounded in remote sensing and aerial surveillance domains, where varying altitudes, sensor angles, and environmental conditions introduce significant geometric variability \cite{zou2023object, Nikouei_2025}.

\subsection{Small Object Detection}
The detection of small objects represents a specialized and pathological sub-domain within computer vision. While general object detection on datasets like MS COCO achieves high performance, accuracy degrades precipitously as object size decreases. This is particularly critical in aerial imagery, where the combination of high altitude and wide field-of-view results in targets of interest—such as vehicles, pedestrians, or small maritime vessels—occupying a minute fraction of the pixel space \cite{liu2020deep, Cheng_2023}.

\textbf{Defining ``Small'' Objects} \\
The definition of a ``small object'' is contingent upon the application context and the dataset standards:
\begin{enumerate}
    \item \textbf{Absolute Scale (MS COCO Definition):} The most widely accepted definition comes from the MS COCO evaluation protocol, which categorizes objects with a spatial area of less than $32 \times 32$ pixels as small ($area < 1024$ pixels) \cite{lin2014microsoft, Cheng_2023}.
    \item \textbf{Relative Scale (SPIE Definition):} The Society of Photo-Optical Instrumentation Engineers (SPIE) defines small objects based on the image coverage ratio. An object is considered small if it occupies less than 0.12\% of the total image area (e.g., roughly $9 \times 9$ pixels in a $256 \times 256$ image) \cite{Nikouei_2025}.
    \item \textbf{Aerial-Specific Scale (AI-TOD):} In specialized aerial datasets like AI-TOD, the definition is even more stringent, with the average object size often being around 12.8 pixels, significantly smaller than the COCO standard \cite{wang2021aitod}.
\end{enumerate}

\textbf{Challenges in Feature Extraction and Representation} \\
The difficulty in detecting small objects stems from fundamental limitations in the architecture of Convolutional Neural Networks (CNNs), specifically related to feature hierarchy and resolution.

\begin{itemize}
    \item \textbf{Feature Vanishing and Dilution:} Deep CNNs rely on successive downsampling operations (strided convolutions or pooling) to increase the Receptive Field (RF) and abstract high-level semantic features. A standard backbone (e.g., ResNet or CSPDarknet) typically has a total stride of 32 ($S=32$). This means an input image of size $640 \times 640$ is reduced to a feature map of $20 \times 20$. Under this transformation, a small object of size $16 \times 16$ pixels in the input is theoretically mapped to an area of $0.5 \times 0.5$ pixels in the final feature map \cite{Nikouei_2025, CHEN2024117194}. In practice, this sub-pixel representation means the object's spatial information is effectively aggregated into a single feature vector mixed with surrounding background information, leading to severe feature dilution or complete vanishing of the signal \cite{rekavandi2023transformerssmallobjectdetection, CHEN2024117194}.
    \item \textbf{Low Signal-to-Noise Ratio (SNR):} Small objects possess very few constituent pixels, limiting the visual information available for the network to learn discriminative features. Unlike large objects that exhibit rich internal textures and clear geometric structures (edges, corners), small objects often appear as amorphous blobs \cite{Nikouei_2025, CHEN2024117194}. This feature scarcity makes them highly susceptible to background clutter; a small rock or a patch of texture can easily be misclassified as a vehicle due to the lack of distinguishing details \cite{YOLO-Air_10980347}.
    \item \textbf{Occlusion and Dense Clustering:} In aerial imagery, small objects often appear in dense clusters (e.g., a parking lot full of cars or a flotilla of ships). The overlap between valid objects and the interference from the background complicates the bounding box regression. Standard Intersection over Union (IoU) metrics are highly sensitive to small positional shifts for small objects; a misalignment of just a few pixels can result in a zero IoU score, destabilizing the training process \cite{Nikouei_2025, yang2020r3detrefinedsinglestagedetector}.
    \item \textbf{Context Dependence:} Because intrinsic features are weak, the detection of small objects relies heavily on contextual cues (e.g., a car is likely on a road, a ship is likely in water). However, standard CNN operations with fixed receptive fields may fail to capture this global context effectively if the object is isolated or the background is heterogeneous \cite{rekavandi2023transformerssmallobjectdetection, CHEN2024117194}.
\end{itemize}

\subsection{Oriented Object Detection}
Traditional object detection systems rely on Horizontal Bounding Boxes (HBB), parameterized by $(x_{min}, y_{min}, x_{max}, y_{max})$ or $(x_c, y_c, w, h)$. While sufficient for ground-level photography where gravity imposes a natural vertical orientation on most objects, HBBs are fundamentally inadequate for aerial and satellite imagery \cite{liu2020deep, ding2019learning}.

\textbf{Geometric Mismatch and the HBB Limitation} 
In top-down aerial views, objects have an additional degree of freedom: rotation. A ship or vehicle can point in any direction, so axis-aligned bounding boxes either include excessive background or misalign the target, which makes IoU scores unreliable even when the localization is correct \cite{ding2019learning, xia2019dotalargescaledatasetobject}. The predominance of arbitrary orientations in aerial data renders Horizontal Bounding Boxes inadequate; instead, oriented rectangles—parameterized by center coordinates, width, height, and rotation angle—are preferred for tight localization.

\subsection{Deformable Convolution}
The output feature $y(p_0)$ at location $p_0$ of a regular convolution is the weighted sum of features sampled at a fixed grid:
\begin{equation}
y(p_0) = \sum_{p_n \in \mathcal{R}} w(p_n) \cdot x(p_0 + p_n) \label{eq:standard_conv}
\end{equation}
where $w(p_n)$ are the kernel weights and $x$ is the input map \cite{lecun2015deep, dai2017deformableconvolutionalnetworks}.

\textbf{Deformable Convolution (DCNv1)} 
Deformable convolution augments the regular grid $\mathcal{R}$ with 2D offsets $\{ \Delta p_n | n=1, \dots, N \}$, where $N = |\mathcal{R}|$. The equation transforms to:
\begin{equation}
y(p_0) = \sum_{p_n \in \mathcal{R}} w(p_n) \cdot x(p_0 + p_n + \Delta p_n)
\end{equation}
Here, the sampling is performed at the irregular locations $p_n + \Delta p_n$. These offsets allow the sampling points to shift to cover the semantic parts of an object—for instance, spreading along the wings of a rotated airplane rather than sampling the empty tarmac \cite{dai2017deformableconvolutionalnetworks}.

\textbf{Bilinear Interpolation} 
Since the learned offsets $\Delta p_n$ are typically fractional (continuous values), the pixel coordinates $p = p_0 + p_n + \Delta p_n$ do not align with the integer grid of the feature map. To compute the pixel value $x(p)$, bilinear interpolation is employed:
\begin{equation}
x(p) = \sum_{q} G(q, p) \cdot x(q)
\end{equation}
where $q$ enumerates the integral spatial locations in the feature map (e.g., the 4 nearest neighbors), and $G(q, p)$ is the bilinear interpolation kernel:
\[
G(q, p) = \max(0, 1 - |q_x - p_x|) \cdot \max(0, 1 - |q_y - p_y|)
\]
This formulation ensures that the operation is fully differentiable, allowing the offsets $\Delta p_n$ to be learned via standard backpropagation \cite{dai2017deformableconvolutionalnetworks}.

\textbf{Modulated Deformable Convolution (DCNv2)} 
While DCNv1 allows the sampling points to move, DCNv2 introduces a \textbf{modulation mechanism} to further enhance feature extraction capability. It adds a learnable scalar weight $\Delta m_k$ to each sampling point, bounded between 0 and 1 via a sigmoid function. The formulation becomes:
\begin{equation}
y(p_0) = \sum_{p_n \in \mathcal{R}} w(p_n) \cdot x(p_0 + p_n + \Delta p_n) \cdot \Delta m_n
\end{equation}
This modulation scalar $\Delta m_n$ allows the network to adjust the ``amplitude'' or importance of each sampling point. Crucially, it enables the network to ``turn off'' sampling points that fall on irrelevant background regions or noise, acting as a local attention mechanism within the convolution kernel.

\subsection{Integration and Offset Learning}
The offsets $\Delta p_n$ and modulation scalars $\Delta m_n$ are not fixed parameters; they are dynamic outputs of the network itself. They are generated by a separate, lightweight convolutional layer (typically a $3 \times 3$ conv) applied to the same input feature map $x$ \cite{dai2017deformableconvolutionalnetworks, zhu2019deformable}.
\begin{itemize}
    \item \textbf{Input:} A feature map $x$ of size $H \times W \times C$ serves as the shared source for both the detection head and the offset generator.
    \item \textbf{Offset Generator:} A convolution layer produces a tensor of size $H \times W \times 3N$ (with $N$ denoting the kernel size, e.g., 9 for $3 \times 3$). The $3N$ channels encode the $x$-offset, $y$-offset, and modulation scalar $m$ for each kernel element, enabling the kernel to adapt to local geometry \cite{dai2017deformableconvolutionalnetworks, zhu2019deformable}.
    \item \textbf{Conditioning:} Because offsets are predicted from the same visual evidence they transform, the network can ``look'' at the object's morphology (e.g., a ship's elongated hull) and emit offsets that align the receptive field with that structure, achieving rotation and scale adaptability.
\end{itemize}

\section{Attention Mechanisms}

\subsection{Fundamentals of Visual Attention}
The concept of ``attention'' in computer vision draws inspiration from human cognitive systems, where the visual cortex selectively focuses on specific parts of a scene to process relevant information while ignoring the rest \cite{vaswani2017attention, xia2022visiontransformerdeformableattention}. In the context of Deep Learning, an attention mechanism is mathematically defined as a dynamic weight adjustment function. It computes a set of ``importance scores'' or weights based on the input features and applies these weights to emphasize informative regions (e.g., objects) and suppress irrelevant ones (e.g., background clutter) \cite{vaswani2017attention}.

Standard attention mechanisms, such as the Self-Attention found in Transformers, compute the relationship between every pair of pixels in the feature map to capture global context.
\begin{equation}
\text{Attention}(Q, K, V) = \text{softmax}\left(\frac{QK^T}{\sqrt{d_k}}\right)V
\end{equation}
While powerful, this global computation has a quadratic complexity of $O(H^2 W^2)$ with respect to the feature map size. For small object detection, which requires high-resolution feature maps to prevent feature vanishing, this computational cost is often prohibitive \cite{zhu2021deformabledetrdeformabletransformers, vaswani2017attention}.

\subsection{Deformable Attention}
Deformable Attention, introduced in Deformable DETR, addresses the computational bottleneck of standard attention by combining the sparse sampling principles of DCN with the relation modeling of Transformers. Instead of attending to all pixels in the image, each query element attends only to a small, fixed set of key sampling points learned from the data.

\textbf{Mathematical Formulation} \\
Given an input feature map $x$, a query element $q$ with content feature $z_q$, and a 2D reference point $p_q$, the Deformable Attention output is computed as:

\begin{equation}
\text{DeformAttn}(z_q, p_q, x) = \sum_{m=1}^{M} W_m \sum_{k=1}^{K} A_{mqk} \cdot W'_m x(p_q + \Delta p_{mqk})
\end{equation}

\begin{itemize}
    \item \textbf{Sparse Sampling ($K$):} $k$ indexes a small set of sampling points (e.g., $K=4$). This reduces the complexity from quadratic $O(H^2 W^2)$ to linear $O(HW \cdot K)$, making it feasible for high-resolution maps needed for small objects \cite{zhu2021deformabledetrdeformabletransformers}.
    \item \textbf{Reference Point ($p_q$):} Unlike standard attention which is location-agnostic, Deformable Attention is grounded at a reference point $p_q$. In the encoder, this is the grid location; in the decoder, it is predicted from object queries \cite{zhu2021deformabledetrdeformabletransformers}.
    \item \textbf{Sampling Offsets ($\Delta p_{mqk}$):} Similar to DCN, the network predicts offsets $\Delta p_{mqk}$ from the query feature $z_q$. This allows the attention head to dynamically ``look'' for information in flexible locations relative to the reference point, adapting to the object's scale and shape \cite{zhu2021deformabledetrdeformabletransformers}.
    \item \textbf{Attention Weights ($A_{mqk}$):} These are scalar weights predicted from $z_q$, normalized via softmax such that $\sum A_{mqk} = 1$. They determine the contribution of each sampled point \cite{zhu2021deformabledetrdeformabletransformers}.
\end{itemize}

\textbf{Synergy with Small Oriented Objects} \\
The Deformable Attention mechanism acts as a \textbf{spatial filter}. By restricting attention to $K$ points, it inherently suppresses the vast majority of background noise pixels that would otherwise contribute to the weighted sum in global attention \cite{zhu2021deformabledetrdeformabletransformers, wang2023internimageexploringlargescalevision}. For small, oriented objects, the learnable offsets $\Delta p_{mqk}$ allow the model to concentrate its limited computational resources (``glances'') precisely on the object's key features (e.g., the bow and stern of a ship), regardless of its orientation, while ignoring the surrounding water or land clutter. This significantly enhances the feature representation of small targets that are otherwise prone to being washed out in standard aggregation operations.
%
% Bab 3 : Perancangan
%-------------------------------------------------------------------------
\chapter{\babTiga}
%--------------------------------------------------------------------------
This thesis proposes new quantum coding scheme 

% Bab 4 : Pengujian
\chapter{\babempat}
This section evaluates the performance of the proposed codes in several aspects (i) syndrome extraction produces unique codes for each single errors occurred in quantum and (ii) the performance of quantum word error rate (QWER). 

% Bab 5 : Kesimpulan dan Saran
%---------------------------------------------------------------
\chapter{\babLima}
%---------------------------------------------------------------
\section{Conclusions}
This thesis has proposed a new quantum coding scheme 

\section{Future Works}
{\color{black}This thesis can be further possible be developed, especially, in terms of QWER performance under the other quantum channels.}


%
\bibliographystyle{IEEEtran}
\bibliography{ref_database}
%\newpage
% Lampiran 
%\begin{appendix}
%	\pagenumbering{arabic}
%	\include{markLampiran}
%%	\input{appendices_IEEE}
%	\input{appendices}
%	%\setcounter{page}{2}
%\end{appendix}
\end{document}