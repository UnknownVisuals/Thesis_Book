%-----------------------------------------------------------------------------%
% Informasi Mengenai Dokumen
%-----------------------------------------------------------------------------%
% 
% Judul laporan. 
\var{\judul}{\setstretch{1}Babibu}
% 
% Tulis kembali judul laporan, kali ini akan diubah menjadi huruf kapital
\Var{\Judul}{\setstretch{1}Deformable Convolution Network and Deformable Attention in YOLO11 for Oriented Small Object Detection}

% Tipe laporan, dapat berisi Skripsi, Tugas Akhir, Thesis, atau Disertasi
\var{\type}{Master Thesis}
% 
% Tulis kembali tipe laporan, kali ini akan diubah menjadi huruf kapital
\Var{\Type}{Master Thesis}
% 

% Tulis nama penulis 
\Var{\Penulis}{Reynaldhi Tryana Graha}
% Tulis NPM penulis
\var{\nim}{201012420030}
\var{\alamat}{Rancasari, Bandung, Jawa Barat}
\var{\tlp}{+6285156964564}
\var{\email}{\textit{reynaldhitryanagraha@student.telkomuniversity.ac.id}}

% Tuliskan Fakultas dimana penulis berada
\Var{\Fakultas}{School of Electrical Engineering}
\var{\fakultas}{School of Electrical Engineering}
% 
% Tuliskan Program Studi yang diambil penulis
\Var{\Program}{Master of Electrical-Telecommunication Engineering}
\var{\program}{Master of Electrical-Telecommunication Engineering}
% 
% Tuliskan tahun publikasi laporan
\Var{\Tahun}{2025}
 
% Tuliskan tanggal pengesahan laporan, waktu dimana laporan diserahkan ke penguji/sekretariat
\var{\tanggalPengesahan}{\today} 

% Tuliskan pembimbing 
\var{\pembimbingSatu}{Dr. Koredianto Usman, S.T., M.Sc.}
\var{\nikSatu}{02750053}
\var{\pembimbingDua}{Suryo Adhi Wibowo, S.T., M.T., Ph.D.}
\var{\nikDua}{1087003}
% 
% Alias untuk memudahkan alur penulisan paa saat menulis laporan
\var{\saya}{Author}

%-----------------------------------------------------------------------------%
% Judul Setiap Bab
%-----------------------------------------------------------------------------%
% 
% Berikut ada judul-judul setiap bab. 
% Silahkan diubah sesuai dengan kebutuhan. 
% 
\var{\kataPengantar}{PREFACE}
\var{\babSatu}{INTRODUCTION}
\var{\babDua}{BASIC CONCEPT}
\var{\babTiga}{SYSTEM DESIGN AND MODEL}
\var{\babempat}{PERFORMANCE EVALUATIONS}
\var{\babLima}{CONCLUSION}
